% !TEX program = xelatex
\documentclass[a4paper,11.5pt,UTF8]{ctexart}
\usepackage{a4}
\usepackage{amsmath, amssymb, amsthm, graphicx, xspace,array,url}
\usepackage{setspace}
\usepackage{float}
\usepackage{listings,color,xcolor,fontspec}

\CTEXsetup[format={\Large\bfseries}]{section}

\title{Theoretical Analysis for 2D Method}
\author{张卓涵 3190101161}
\usepackage[a4paper,left=20mm,right=20mm,top=30mm,bottom=20mm]{geometry}

\newtheorem{example}{例}[section]
\newtheorem*{remark}{注}
\newtheorem{thm}{定理}[section]
\newtheorem{lemma}{引理}[section]
\newtheorem{define}{定义}[section]
\newtheorem{corollary}{推论}[section]

\begin{document}
\begin{figure}[t]
\begin{minipage}[h]{0.25\linewidth}
	\includegraphics[width=4.0cm]{./figure/ZJU2.jpeg}
\end{minipage}
\hfill
\begin{minipage}[h]{.7\linewidth}
	\begin{flushright}
			\Large{微分方程数值解
				\vspace{3mm}	\\
				   2022 夏
				\vspace{3mm}	\\
				   张卓涵 \hspace{3mm}3190101161}
	\end{flushright}
\end{minipage}
\rule{\linewidth}{0.1em}
\end{figure}
\begin{center}
	\huge{\textbf{Theoretical Analysis for 2D Method}}
\end{center}
\begin{large}
	
\section{Question}
\par 考虑二维Dirichlet边值条件的BVP:
\begin{equation}
	\begin{cases}
		-\Delta u = f, & in\ \Omega:=(0,1)^2 \\
		u = 0, & on\ \partial\Omega
	\end{cases}
	\label{BVP}
\end{equation}

	
\section{Relaxation}
\begin{thm}
	二维BVP求解中的线性方程组系数矩阵$A_{2D} = I\otimes A + A\otimes I$,其中$A$是一维Dirichlet边值条件BVP线性系统的系数矩阵.并且$A_{2D}$的特征值和特征向量是
	$$\lambda_{ij}=\lambda_i+\lambda_j,\quad \mathbf{W}_{ij}=\mathbf{w}_j\otimes\mathbf{w}_i$$
	其中$\lambda_i$和$\lambda_j$是A的特征值,$\mathbf{w}_i$和$\mathbf{w}_j$是对应的特征向量.
\end{thm}
\begin{proof}
	这直接由讲义中的Theorem 7.48和Theorem 7.51得到.
\end{proof}

\begin{thm}
	\eqref{BVP}对应的线性系统的加权Jacobi迭代矩阵是
	$$T_{\omega} = (1-\omega)I+\omega D^{-1}(L+U) = I-\frac{\omega h^2}{4}A_{2D}$$
	其特征向量就是$A_{2D}$的特征向量,对应的特征值是
	$$\lambda_{ij} = 1-\omega\sin^2(\frac{i\pi}{2n})-\omega\sin^2(\frac{j\pi}{2n})$$
\end{thm}
\begin{proof}
	这由讲义Lemma 9.16直接得到.
\end{proof}

\section{Restriction and prolongation}
\begin{lemma}
	二维网格的full weighting算子和linear算子满足
	$$I_{h(2D)}^{2h} = I^{2h}_{h}\otimes I^{2h}_{h},\quad I_{2h(2D)}^{h}=I^{h}_{2h}\otimes I^{h}_{2h}$$
	其中$I_{h}^{2h}$和$I_{2h}^{h}$分别时一维的full weighting算子和linear算子.
\end{lemma}
\begin{example}
	考虑在$n=3$的情况,我们有二维full weighting算子为
	\begin{align*}
		I^{2h}_{h(2D)} &= \frac{1}{16}\begin{bmatrix}
			-1 & 2 & -1 & -2 & 4 & -2 & -1 & 2 & -1 
		\end{bmatrix} \\
		&= \frac{1}{4}\begin{bmatrix} -1 & 2 & -1 \end{bmatrix} \otimes
		   \frac{1}{4}\begin{bmatrix} -1 & 2 & -1 \end{bmatrix} \\
		&= I^{2h}_{h}\otimes I^{2h}_{h}
	\end{align*}
\end{example}

\section{Two-grid correction}
\begin{lemma}
	二维的Two-grid correction作用在误差向量上的迭代矩阵是
	$$TG=T_{\omega}^{\nu_2}[I_{2D}-I^{h}_{2h(2D)}(A^{2h}_{2D})^{-1}I^{2h}_{h(2D)}A^h_{2D}]T_{\omega}^{\nu_1}$$
\end{lemma}
\begin{proof}
	这由讲义Lemma 9.31推广得到.
\end{proof}

\section{The spectral picture}
\begin{lemma}
	full weighting restriction算子作用在$A_{2D}^{h}$的特征向量$\mathbf{W}_{ij}^{h}$上得到
	\begin{align*}
		I^{2h}_{h(2D)}\mathbf{W}_{ij}^{h} := c_ic_j\mathbf{W}_{ij}^{2h} = \cos^2\frac{i\pi}{2n}\cos^2\frac{j\pi}{2n}\mathbf{W}_{ij}^{2h}
	\end{align*}
\end{lemma}
\begin{proof}
	\begin{align*}
		I^{2h}_{h(2D)}\mathbf{W}_{ij}^{h} &= (I^{2h}_h\otimes I^{2h}_h)\cdot(\mathbf{w}_j^h\otimes\mathbf{w}_i^h) \\
		&= (I^{2h}_{h}\mathbf{w}_{j}^{h})\otimes(I^{2h}_{h}\mathbf{w}_{i}^{h}) \\
		&= (c_i\mathbf{w}_{j}^{2h})\otimes(c_j\mathbf{w}_{i}^{2h}) \\
		&= c_ic_j\mathbf{W}_{ij}^{2h}
	\end{align*}
	这里第三步是由讲义Lemma 9.38得到.
\end{proof}

\begin{lemma}
	linear interpolation算子作用在$A_{2D}^{h}$的特征向量$\mathbf{W}_{ij}^{h}$上得到
	$$I^h_{2h(2D)}\mathbf{W}_{ij}^{2h} = c_ic_j\mathbf{W}_{ij}^h-c_is_j\mathbf{W}_{ij'}^h-c_js_i\mathbf{W}_{i'j}^h+s_is_j\mathbf{W}_{i'j'}^h$$
	其中,$c_i = \cos^2\frac{i\pi}{2n}$,$s_i=\sin^2\frac{i\pi}{2n}$,$i'=n-i$,$j'=n-j$.
\end{lemma}
\begin{proof}
	\begin{align*}
		I^h_{2h(2D)}\mathbf{W}_{ij}^{2h} &= (I^h_{2h}\otimes I^h_{2h})\cdot(\mathbf{w}_{j}^{2h}\otimes\mathbf{w}_{i}^{2h}) \\
		&= (I^h_{2h}\mathbf{w}_{j}^{2h})\otimes(I^h_{2h}\mathbf{w}_{i}^{2h}) \\
		&= (c_j\mathbf{w}_{j}^{2h}-s_j\mathbf{w}_{j'}^h)\otimes(c_i\mathbf{w}_{i}^{2h}-s_i\mathbf{w}_{i'}^h) \\
		&= c_ic_j\mathbf{W}_{ij}^h-c_is_j\mathbf{W}_{ij'}^h-c_js_i\mathbf{W}_{i'j}^h+s_is_j\mathbf{W}_{i'j'}^h
	\end{align*}
	这里第三个等号由讲义Lemma 9.39得到.
\end{proof}

\begin{thm}
	Two-grid correction作用在子空间$span\{\mathbf{W}_{ij}^{h},\mathbf{W}_{i'j}^{h},\mathbf{W}_{ij'}^{h},\mathbf{W}_{i'j'}^{h}\}$上是不变的,且
	\begin{align*}
		TG\mathbf{W}_{ij}^{h} &= \lambda_{ij}^{\nu_1+\nu_2}\left(1-\frac{\chi_ic_j+\chi_jc_i}{\chi_i+\chi_j}c_ic_j\right)\mathbf{W}_{ij} + \lambda_{ij}^{\nu_1}\lambda_{ij'}^{\nu_2}\frac{\chi_ic_j+\chi_jc_i}{\chi_i+\chi_j}c_is_j\mathbf{W}_{ij'}^{h} \\ &+ \lambda_{ij}^{\nu_1}\lambda_{i'j}^{\nu_2}\frac{\chi_ic_j+\chi_jc_i}{\chi_i+\chi_j}c_js_i\mathbf{W}_{i'j}^{h} - \lambda_{ij}^{\nu_1}\lambda_{i'j'}^{\nu_2}\frac{\chi_ic_j+\chi_jc_i}{\chi_i+\chi_j}s_is_j\mathbf{W}_{i'j'}^{h}
	\end{align*}
	其中,$\lambda_{ij}=1-\omega\sin^2(\frac{i\pi}{2n})-\omega\sin^2(\frac{j\pi}{2n})$是$T_{\omega}$的特征值,$\chi_i=\sin^2\frac{i\pi}{n}$,$\chi_j=\sin^2\frac{j\pi}{n}$.
\end{thm}
\begin{proof}
	首先不考虑relaxation过程,即假设$\nu_1=\nu_2=0$,
	\begin{align*}
		&A^h_{2D}\mathbf{W}_{ij}^h=\frac{4}{h^2}(s_i+s_j)\mathbf{W}_{ij}^h \\
		\implies& I^{2h}_{h(2D)}A^h_{2D}\mathbf{W}_{ij}^h=\frac{4}{h^2}(s_i+s_j)c_ic_j\mathbf{W}_{ij}^{2h} \\
		\implies& (A^{2h}_{2D})^{-1}I^{2h}_{h(2D)}A^h_{2D}\mathbf{W}_{ij}^h=\frac{h^2}{\chi_i+\chi_j}\frac{4}{h^2}(s_i+s_j)c_ic_j\mathbf{W}_{ij}^{2h}=\frac{\chi_ic_j+\chi_jc_i}{\chi_i+\chi_j}\mathbf{W}_{ij}^{2h} \\
		\implies& I_{2h(2D)}^{h}(A^{2h}_{2D})^{-1}I^{2h}_{h(2D)}A^h_{2D}\mathbf{W}_{ij}^h=\frac{\chi_ic_j+\chi_jc_i}{\chi_i+\chi_j}[c_ic_j\mathbf{W}_{ij}^h-c_is_j\mathbf{W}_{ij'}^h-c_js_i\mathbf{W}_{i'j}^h+s_is_j\mathbf{W}_{i'j'}^h] \\
		\implies& [I-I_{2h(2D)}^{h}(A^{2h}_{2D})^{-1}I^{2h}_{h(2D)}A^h_{2D}]\mathbf{W}_{ij}^h=\left(1-\frac{\chi_ic_j+\chi_jc_i}{\chi_i+\chi_j}c_ic_j\right)\mathbf{W}_{ij}^h \\
		&+ \frac{\chi_ic_j+\chi_jc_i}{\chi_i+\chi_j}c_is_j\mathbf{W}_{ij'}^{h} + \frac{\chi_ic_j+\chi_jc_i}{\chi_i+\chi_j}c_js_i\mathbf{W}_{i'j}^{h} - \frac{\chi_ic_j+\chi_jc_i}{\chi_i+\chi_j}s_is_j\mathbf{W}_{i'j'}^{h}
	\end{align*}
	其中第二步和第四步来自上述两个引理,考虑上smoothing的过程得到:
	\begin{align*}
	TG\mathbf{W}_{ij}^{h} &= \lambda_{ij}^{\nu_1+\nu_2}\left(1-\frac{\chi_ic_j+\chi_jc_i}{\chi_i+\chi_j}c_ic_j\right)\mathbf{W}_{ij}^h + \lambda_{ij}^{\nu_1}\lambda_{ij'}^{\nu_2}\frac{\chi_ic_j+\chi_jc_i}{\chi_i+\chi_j}c_is_j\mathbf{W}_{ij'}^{h} \\ &+ \lambda_{ij}^{\nu_1}\lambda_{i'j}^{\nu_2}\frac{\chi_ic_j+\chi_jc_i}{\chi_i+\chi_j}c_js_i\mathbf{W}_{i'j}^{h} - \lambda_{ij}^{\nu_1}\lambda_{i'j'}^{\nu_2}\frac{\chi_ic_j+\chi_jc_i}{\chi_i+\chi_j}s_is_j\mathbf{W}_{i'j'}^{h}
	\end{align*}
\end{proof}
\begin{remark}
	对$TG\mathbf{W}_{ij}^h$中$\mathbf{W}_{ij}^h$,$\mathbf{W}_{ij'}^h$,$\mathbf{W}_{i'j}^h$,$\mathbf{W}_{i'j'}^h$四项的系数讨论知,四个系数均充分小于$1$.
\end{remark}

\begin{corollary}
	类似上述讨论,$TG$作用在$\mathbf{W}_{ij'}^h$,$\mathbf{W}_{i'j}^h$,$\mathbf{W}_{i'j'}^h$上也有四项系数充分小于$1$,即:
	\begin{align*}
	TG\begin{bmatrix}
		\mathbf{W}_{ij}^h \\
		\mathbf{W}_{ij'}^h \\
		\mathbf{W}_{i'j}^h \\
		\mathbf{W}_{i'j'}^h
	\end{bmatrix}=\begin{bmatrix}
		k_{11} & k_{12} & k_{13} & k_{14} \\
		k_{21} & k_{22} & k_{23} & k_{24} \\
		k_{31} & k_{32} & k_{33} & k_{34} \\
		k_{41} & k_{42} & k_{43} & k_{44} 
	\end{bmatrix} \begin{bmatrix}
		\mathbf{W}_{ij}^h \\
		\mathbf{W}_{ij'}^h \\
		\mathbf{W}_{i'j}^h \\
		\mathbf{W}_{i'j'}^h
	\end{bmatrix}
	\end{align*}
	其中,$k_{ij}$都是充分小于$1$的数.
\end{corollary}
综上,便说明了二维多重网格方法的收敛性.





\end{large}
\end{document}